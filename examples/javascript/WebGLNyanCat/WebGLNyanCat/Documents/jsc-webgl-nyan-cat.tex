% TexWorks
% part chapter  section subsection
\documentclass[12pt,leqno]{book}

\usepackage{url}
\usepackage{hyperref}

\usepackage{graphicx}

% \includegraphics

\newcommand{\webdev}{Microsoft Visual Web Developer 2010 Express}
\newcommand{\png}[1]{\includegraphics[type=png,ext=.png,read=.png,width=\textwidth]{#1}}
\newcommand{\figpng}[2]{\begin{figure}[htb]\centering\png{#1}\caption{#2}\end{figure}}

\usepackage{framed}


\title{WebGL Nyan Cat}
\author{\includegraphics{../Design/Preview.png} \\ \\ \\ Arvo Sulakatko \\ \\ \href{http://www.jsc-solutions.net}{jsc-solutions.net} }

\begin{document}

\maketitle



\tableofcontents
\listoffigures

% why what how

\chapter{The Why}
\marginpar{[...] people don't buy what you do. people buy why you do it!}

% \figpng{Images/www.youtube.comvhKksAVmekAE_-_Google_Chrome-2012-03-14_16.27.22}
% {Physical Spider To Be Programmed}

\subsection{Creating a preview image}

See also: http://dl.dropbox.com/u/6213850/WebGL/nyanCat/nyan.html


\subsection{Setting up TeXworks for LaTex documentation}

\webdev needs some additional configuration. While creating new commands in latex one needs to remember they cannot have numbers in the name.


\chapter{References}

\subsection{Document Source}

\url{https://jsc.svn.sourceforge.net/svnroot/jsc/examples/javascript/ArduinoSpiderControlCenter/SpiderModel/Documents/spider.tex}

\subsection{Project Source}
\url{https://jsc.svn.sourceforge.net/svnroot/jsc/examples/javascript/ArduinoSpiderControlCenter/SpiderModel/}



\subsection{Video}
\url{http://www.youtube.com/v/hKksAVmekAE}

\subsection{JSC Web Installer}
\url{http://download.jsc-solutions.net}

\subsection{Website}
\url{http://www.jsc-solutions.net}

\subsection{Blog}
\url{http://zproxy.wordpress.com}

\end{document}