% TexWorks
% part chapter  section subsection
\documentclass[12pt,leqno]{book}

\usepackage{url}
\usepackage{hyperref}

\usepackage{graphicx}

% \includegraphics

\newcommand{\webdev}{Microsoft Visual Web Developer 2010 Express}
\newcommand{\png}[1]{\includegraphics[type=png,ext=.png,read=.png,width=\textwidth]{#1}}
\newcommand{\figpng}[2]{\begin{figure}[htb]\centering\png{#1}\caption{#2}\end{figure}}

\usepackage{framed}


\title{WebGL Nyan Cat}
\author{\includegraphics{../Design/Preview.png} \\ \\ \\ Arvo Sulakatko \\ \\ \href{http://www.jsc-solutions.net}{jsc-solutions.net} }

\begin{document}

\maketitle



\tableofcontents
\listoffigures

% why what how

\chapter{The Beginning}




\subsection{Creating a new Web Application}

Let's create a new project "WebGL Nyan Cat".

\figpng{Images/Add_New_Item_-_WebGLNyanCat-2012-03-31_12.14.47}
{New Project}




\subsection{Creating a new preview image}

In this document we shall  port the example at http://dl.dropbox.com/u/6213850/WebGL/nyanCat/nyan.html to CSharp.




\subsection{Setting up TeXworks for LaTex documentation}

\webdev needs some additional configuration. While creating new commands in latex one needs to remember they cannot have numbers in the name.

\figpng{Images/jsc-webgl-nyan-cat.pdf_-_TeXworks-2012-03-31_12.31.15}
{TeXworks}



\subsection{Prebuild event}
Just when our PDF is being generated we are ready to test run the project.
First we should do a project rebuild. This is where JSC will generate the AssetsLibrary during the prebuild event.

\figpng{Images/WebGLNyanCat_-_Microsoft_Visual_Web_Developer_2010_Express-2012-03-31_12.37.24}
{Solution Explorer}

At this time the most important type generated for us is the WebGLNyanCat.HTML.Pages.DefaultPage type. Essentially it will give us a typed access to all HTML elements within the HTML document with id attributes. We may however choose to ignore it and just attach to document instead.

\figpng{Images/eXplore-2012-03-31_12.41.39}
{JSC eXplore - WebGLNyanCat.AssetsLibrary}


\subsection{Start Debugging}
Now we should test JSC. Without changing anything else let's just run the project. A web broswer will be started once the build is ready.



\figpng{Images/A_string_from_JavaScript._-_Google_Chrome-2012-03-31_13.00.11}
{Application running inside Chrome}


\subsection{Inspecting the original}
Now we are ready to start. JSC does not only support looking at .NET assemblies. It also supports looking at Web Applications. At this time however the HTML parser is unforgiving and does not help us. We need to do manual code review of the original Web Application to replicate it. Reading the source code reveals it makes use of of some JavaScript libraries, some of which we may choose to ignore. There are also two audio files. No explicit shaders.

\figpng{Images/eXplore-2012-03-31_13.12.27}
{JSC eXplore - Web Application}


\subsection{Preparing the assets}
We need to add the Three.js, nyanlooped.mp3 and nyanslow.mp3 to our project to be dynamically loaded on demand while the application is running. Note that at this time we need to define audio elements in our HTML page to make it explicit that they are needed.

\figpng{Images/WebGLNyanCat_-_Microsoft_Visual_Web_Developer_2010_Express-2012-03-31_13.24.27}
{Solution Explorer - New Assets}

\figpng{Images/eXplore-2012-03-31_13.48.45}
{JSC eXplore - New Assets}

\figpng{Images/A_string_from_JavaScript._-_Google_Chrome-2012-03-31_13.54.10}
{Application running inside Chrome with audio}

\subsection{Implement InitializeContent}
By now we are ready to port the application code. The inline JavaScript shall be rewritten to C# and into InitializeContent.

First thing we do is we comment out everything and put them into named region for outlining. We have to reorder statements such as the init function call actually happens after the function has been defined. The first function to convert will be mouse down which will switch between songs.


\subsection{Adding fullscreen and dispose}
xxx




\chapter{Notes for future work}

\subsection{Less references}
Future versions of JSC shall consider merging ScriptCoreLibr assemblies to keep the number low.

\subsection{Exclude from Project Template}
Some files should be excluded from project template. 

\subsection{Implicit preload of js and css}
At this time we have to dynamically and explicitly load css and js. In the future we should
be able to do that implicitly either if they were defined inline or only linked from HTML.

\subsection{ScriptCoreLib to embrace SystemAction}
Due to legacy code the DOM does not make use of System.Action. It should. In future builds ScriptCoreLib will be changed to make use of Action instead.






\chapter{References}

\subsection{Document Source}

\url{https://jsc.svn.sourceforge.net/svnroot/jsc/examples/javascript/ArduinoSpiderControlCenter/SpiderModel/Documents/spider.tex}

\subsection{Project Source}
\url{https://jsc.svn.sourceforge.net/svnroot/jsc/examples/javascript/ArduinoSpiderControlCenter/SpiderModel/}



\subsection{Video}
\url{http://www.youtube.com/v/hKksAVmekAE}

\subsection{JSC Web Installer}
\url{http://download.jsc-solutions.net}

\subsection{Website}
\url{http://www.jsc-solutions.net}

\subsection{Blog}
\url{http://zproxy.wordpress.com}

\end{document}