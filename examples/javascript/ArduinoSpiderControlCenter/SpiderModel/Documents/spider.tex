% TexWorks
% part chapter  section subsection
\documentclass[12pt,leqno]{book}

\usepackage{url}
\usepackage{hyperref}

\usepackage{graphicx}

% \includegraphics

\newcommand{\png}[1]{\includegraphics[type=png,ext=.png,read=.png,width=\textwidth]{#1}}
\newcommand{\figpng}[2]{\begin{figure}[H!tb]\centering\png{#1}\caption{#2}\end{figure}}

\usepackage{framed}


\title{WebGL Spider}
\author{\includegraphics{../Design/Preview.png} \\ \\ \\ Arvo Sulakatko \\ \\ \href{http://www.jsc-solutions.net}{jsc-solutions.net} }

\begin{document}

\maketitle



\tableofcontents
\listoffigures

% why what how

\chapter{The Why}
\marginpar{[...] people don't buy what you do. people buy why you do it!}

\figpng{Images/www.youtube.comvhKksAVmekAE_-_Google_Chrome-2012-03-14_16.27.22}
{Physical Spider To Be Programmed}




\section{Intro}

In 2011 I took a course. It was the \textbf{Advanced Topics in Biomechanics} course by \textbf{Adriano Cavalcanti, Ph.D}. During this course we had to come up with various 3D visualizations of different models. I chose to do that within WebGL. As a final task we had to come up with a mechanic spider. My part was to make it move. I had never programmed a robot before. 

I was given a piece of hardware which had a few sensors and four legs.

\figpng{Images/parts}
{What can we see on the spider}




\section{Goal}
For every project to be succesful a goal needs to be set. 
Avoid obstacles
Go to thee light
Stop when there






\chapter{The What - Create a WebGL Spider}
\section{3D Visualization}

\marginpar{Refer to the next section to install jsc!}

In this chapter we shall have a look at how to build on this example on your machine.



\figpng{Images/New_Project-2012-03-13_09.19.10}
{Visual Studio Web Developer Express - New Project}

\figpng{Images/SpiderModel_-_Microsoft_Visual_Studio-2012-03-14_15.39.19}
{Solution Explorer}

\figpng{Images/A_string_from_JavaScript._-_Google_Chrome-2012-03-14_16.12.04}
{Program 23}

\figpng{Images/aboutgpu_-_Google_Chrome-2012-03-14_17.56.30}
{Make sure your device is supprting WebGL}

\begin{framed}
chrome.exe -enable-webgl -enable-apps -ignore-gpu-blacklist
\end{framed}





\section{Arduino}
Although this document briefly describes Arduino related development it is considered out of scope and is not part of the default \textbf{jsc eXperience}.

At this time jsc does not support any languages that target Arduino platform. As such I had to make use of Arduino programming language. Otherwise I could of had my CSharp code compiled to Arduino. This would of had allowed me to use the same code in the visualization and on the chip.

\figpng{Images/spiderwalk2__Arduino_1.0-2012-03-14_18.04.43}
{Arduino - spiderwalk2.ino}


\subsection{Lessons learned}
While programming for Arduino I had to manually port my code I had written for the visualizer to the Arduino platform. In doing so I discovered that the int is considered to be 16 bits and that I cannot make use of function pointers. To overcome that I had to divide before I did my multiplication. Yes I had to track down an overflow bug before I realized this. The callbacks I used were simple. They only had a few parameters. This allowed me to replace the function pointer with pointer to variable and have the same behaviour of code.




\chapter{The How - Install JSC}

\figpng{Images/jsc_-_Google_Chrome-2012-03-14_14.33.04}
{Download JSC at \url{http://download.jsc-solutions.net}}

Installing JSC is easy. Before you do make sure you have installed Visual Studio 2010 Web Developer Express.




\chapter{References}

\url{https://jsc.svn.sourceforge.net/svnroot/jsc/examples/javascript/ArduinoSpiderControlCenter/SpiderModel/Documents/spider.tex}


Source
Video
Installer
Website
Blog

\end{document}