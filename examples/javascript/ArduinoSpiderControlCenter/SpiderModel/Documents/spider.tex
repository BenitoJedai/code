% TexWorks
% part chapter  section subsection
\documentclass[12pt,leqno]{book}

\usepackage{graphicx}

% \includegraphics

\newcommand{\png}[1]{\includegraphics[type=png,ext=.png,read=.png,width=\textwidth]{#1}}
\newcommand{\figpng}[2]{\begin{figure}[htb]\centering\png{#1}\caption{#2}\end{figure}}

\usepackage{framed}


\title{WebGL Spider}
\author{Arvo Sulakatko}

\begin{document}

\maketitle

\tableofcontents
\listoffigures

% why what how
\chapter{Why}



\marginpar{[...] people don't buy what you do. people buy why you do it!}
In 2011 I took a course. It was a course on biomechanics. During this course we had to come up with 3D visualizations of different models. As a final task we had to come up with a mechanic spider. My part was to make it move. I had never programmed a robot before. 

\chapter{What - Create a WebGL Spider}
In this chapter we shall have a look at how to build on this example on your machine.

\begin{framed}
!!! For those of you who are new to JSC please refer to the next section to install it.
\end{framed}


\figpng{Images/New_Project-2012-03-13_09.19.10}
{Visual Studio Web Developer Express - New Project}

\chapter{How - Install JSC}

get jsc

\figpng{Images/jsc_-_Google_Chrome-2012-03-14_14.33.04}
{Download JSC at http://download.jsc-solutions.net}


\end{document}